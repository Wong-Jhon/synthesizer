\documentclass[12pt]{article}
\usepackage{amssymb, amsthm, graphics, graphicx}
\usepackage{amsmath}

%these are initial settings. Ron recommended them, so that's just what I use.
\textwidth = 6.5 in
\textheight = 9 in
\oddsidemargin = 0.0 in
\evensidemargin = 0.0 in
\topmargin = -.75in
\parskip = 0.2 in
\parindent = 15 pt
\renewcommand{\baselinestretch}{1.25}

\providecommand{\e}[1]{\ensuremath{\times 10^{#1}}}
\providecommand{\degree}[0]{\ensuremath{^{\circ}}}

\providecommand{\ah}[1]{\ensuremath{\hat{a}_{#1}}}
\providecommand{\ahx}[0]{\ensuremath{\ah{x}}}
\providecommand{\ahy}[0]{\ensuremath{\ah{y}}}
\providecommand{\ahz}[0]{\ensuremath{\ah{z}}}
\providecommand{\pfs}[0]{\ensuremath{\epsilon_{0}}} %permittivity of free space => 1/(36\pi) \e{-9}
%\providecommand{\vec}[1]{\ensuremath{\textbf{#1}}} %vector notation ACTUALLY \vec is _real_ vector notation
\providecommand{\ohm}[0]{\ensuremath{\Omega}}


\newtheorem*{prob}{Problem}

\begin{document}
\begin{flushright}
\textbf{Charles Julian Knight}\\
ECE4892\\
\today
\end{flushright}


\begin{center}
\huge HW 2
\end{center}

\begin{prob}[1.a]{
When the output of the comparator is $+10$ V, what is the voltage at the positive input terminal of U3A?
}\end{prob}
We can assume D1 is off, and D2 is on. Thus the voltage at the $k$ side of D2 is $10-.7=9.3$V. It see's a path to ground through a 100K, a 50K (half of R25), and a 1K. The input voltage is a divider:
\[ V_{in} = 9.3 \frac{1K}{151K} = 61.6 mV \]

\begin{prob}[1.b]{
When the output of the comparator is $+10$ V, using the nonlinear ``tanh'' model for OTA behavior, what is the output current of the OTA as a function of the current control input (pin 1) of the OTA? (Note that unlike the Buchla 259 VCO circuit we looked at in class, the OTA here does not seem to be fully saturated.)
}\end{prob}
The non-linear model:
\[ V_o = 19.2 I_{con} \tanh(v_+ - v_-) = 19.2 I_{con} \tanh(61.6mV) = 1.81 A^{-1}I_{con} \]

\begin{prob}[1.c]{
When the output of the comparator is $+10$ V, what voltage at the output of the integrating op amp (pin 1 of U2-A) would cause 0 V to appear at the positive terminal of the comparator op amp (pin 3 of U4-A)? (Note that this will tell you the maximum level of the triangle wave).
}\end{prob}
We know the output of the comparator to be $10V$, so we can think of the input of the comparator to be a voltage divider with the top fixed and the bottom ($v_{tri}$) variable. I'm calling the input of the comparator $v_{ci}$.
\[ v_{ci} = (10-v_{tri})\frac{120K}{120K+270K} \]
\[ v_{tri} = 10 - v_{ci}\frac{13}{4} \]
Thus, when $v_{ci}=0V$, $v_{tri}=0V$.

\begin{prob}[1.d]{
What is the frequency of the triangle wave as a function of the current control input (pin 1) of the OTA?
}\end{prob}
We're running the output of the OTA into an integrator, so let's integrate our equation from part b:
\[v_{tri} = -\int_0^t 1.81 A^{-1}I_{con} d \tau  = -1.81 A^{-1}I_{con} t \]
We found our voltage swing to be from $61.6$mV to $0$V, and that's the change in one period, so
\[ t = \frac{-61.6mV}{-1.81 A^{-1}I_{con}}\]

Of course, frequency is $1/t$, so
\[ freq = 29.4 \ohm^{-1} I_{con} \]

But since this is one-half swing, the total triangle frequency would be twice this.
Units of seconds should be somewhere in there, not sure where they went. But the proportions should be right.

\begin{prob}[1.e]{
Take a look at the TRI output in the middle of the page (pin 2 of R15). What is the output impedance of the TRI output?
}\end{prob}
The output impedance is simply R15 $= 1K\ohm$.

%%%%%%%%%%%%%%%%%%%%%%%%%%%%%%%%%%%%%%%%%%%%%%%%%%%%%%%%%%%%%%%%%%%%%%%%%%%%%%%%%%%%%%%%%%%%%%
\begin{prob}[2.a]{
What is $I_{ref}$?
}\end{prob}
If $I_{ref}$ is the right-side current, then that opamp is holding the top of that 330K (collector of right BJT) at a virtual ground. The current (from top to bottom) is then
\[ \frac{0V--15V}{330K} = .0455 mA \]

\begin{prob}[2.b]{
 Assuming that the CV offset trim pot is set all the way to the ``right'' (i.e. at ground), what change in CV will cause the control current to double? (Assume the PNP BJTs draw insignificant current through their bases).
}\end{prob}
To get the control current to double, the factor inside the exponent must increase by $\ln(2)$.

The base voltage into the right BJT is given by the voltage divider
\[ v_{rb} = CV\frac{100K||4.7K}{150K+100K||4.7K} = .029*CV \]

So
\[.029CV_2 = .029CV_1 + \ln(2) \]
\[CV_2-CV_1 = \frac{\ln(2)}{.029} = 23.9 V\]
This is stupidly large, so obviously I did something wrong. We want a really small swing in CV, not a really big one.

\begin{prob}[2.c]{
 Assuming the OTA is operating in the linear region, give an expression relating the audio output voltage to the audio input voltage in terms of the current at the control input pin of the OTA. (You may ignore the offset trimming circuitry of the OTA. Assume the positive input of the 3080 is grounded.)
}\end{prob}


\begin{prob}[2.d]{
What is the input impedance of this VCA?
}\end{prob}
\[ 100K + 220 = 100,220 \approx 100K\ohm\]

\begin{prob}[2.e]{
What is the output impedance of this VCA? (It might be ``0'' - remember we're assuming ideal op amps.)
}\end{prob}
The output is buffered by an opamp, so the output impedance is appreciably $0$.
%%%%%%%%%%%%%%%%%%%%%%%%%%%%%%%%%%%%%%%%%%%%%%%%%%%%%%%%%%%%%%%%%%%%%%%%%%%%%%%%%%%%%%%%%%%%%%
\begin{prob}[3.a]{
Given Ray's description of the circuit operation, find the frequency of the oscillator in Hertz in terms of $I_{freq}$. (To make things easy, assume the reset time is finite.)
}\end{prob}

\begin{prob}[3.b]{
Given you result in part (a), what value of $I_{freq}$ would generate a 440 Hz tone?
}\end{prob}

\begin{prob}[3.c]{
Now let's get some practice in reasoning with tempco resistors (see your notes from class session 7). Suppose that R8, R10, R18, R23, R27 aren't there, and we'll focus just on the CV1 input through R15. What is the output of U1-A (pin 1) as a function of voltage CV1 if the tempco is at a temperature of 25 degrees celcius (the base resistance is 2K for 25 degrees celcius)?
}\end{prob}

\begin{prob}[3.d]{
Now suppose you're using Ray's VCO circuit to make sound for an art installation at the Burning Man Project, which can get up to and above 100 degrees fahrenheit during the day. Redo problem (c), except use a temperature of 40 degrees celcius instead of 25 degrees celcius.
}\end{prob}

\end{document}
